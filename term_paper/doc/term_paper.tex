%%%%%%%%%%%%%%%%%%%%%%%%%%%%%%%%%%%%%%%%%
% Journal Article
% Data Warehouse and Mining
% Term Paper Data Transformation
%
% web-guide : https://sites.google.com/a/nirmauni.ac.in/3cs1e23---data-warehousing-and-mining/home/academic-docs/list-of-practicals/practical---1?pli=1
% Gahan M. Saraiya
% 18MCEC10
%
%%%%%%%%%%%%%%%%%%%%%%%%%%%%%%%%%%%%%%%%%
%----------------------------------------------------------------------------------------
%       PACKAGES AND OTHER DOCUMENT CONFIGURATIONS
%----------------------------------------------------------------------------------------
\documentclass[paper=letter, fontsize=12pt]{article}
\usepackage[english]{babel} % English language/hyphenation
\usepackage{amsmath,amsfonts,amsthm} % Math packages
\usepackage[utf8]{inputenc}
\usepackage{xcolor}
\usepackage{float}
\usepackage{lipsum} % Package to generate dummy text throughout this template
\usepackage{blindtext}
\usepackage{graphicx} 
\usepackage{caption}
\usepackage{subcaption}
\usepackage[sc]{mathpazo} % Use the Palatino font
\usepackage[T1]{fontenc} % Use 8-bit encoding that has 256 glyphs
\usepackage{bbding}  % to use custom itemize font
\linespread{1.05} % Line spacing - Palatino needs more space between lines
\usepackage{microtype} % Slightly tweak font spacing for aesthetics
\usepackage[hmarginratio=1:1,top=32mm,columnsep=20pt]{geometry} % Document margins
\usepackage{multicol} % Used for the two-column layout of the document
%\usepackage[hang, small,labelfont=bf,up,textfont=it,up]{caption} % Custom captions under/above floats in tables or figures
\usepackage{booktabs} % Horizontal rules in tables
\usepackage{float} % Required for tables and figures in the multi-column environment - they need to be placed in specific locations with the [H] (e.g. \begin{table}[H])
\usepackage{hyperref} % For hyperlinks in the PDF
\usepackage{lettrine} % The lettrine is the first enlarged letter at the beginning of the text
\usepackage{paralist} % Used for the compactitem environment which makes bullet points with less space between them
\usepackage{abstract} % Allows abstract customization
\renewcommand{\abstractnamefont}{\normalfont\bfseries} % Set the "Abstract" text to bold
\renewcommand{\abstracttextfont}{\normalfont\small\itshape} % Set the abstract itself to small italic text
\usepackage{titlesec} % Allows customization of titles

\usepackage{makecell}
\usepackage{longtable}
\renewcommand\thesection{\Roman{section}} % Roman numerals for the sections
\renewcommand\thesubsection{\Roman{subsection}} % Roman numerals for subsections
%----------------------------------------------------------------------------------------
%       DATE FORMAT
%----------------------------------------------------------------------------------------
\usepackage{datetime}
\newdateformat{monthyeardate}{\monthname[\THEMONTH], \THEYEAR}
%----------------------------------------------------------------------------------------

\titleformat{\section}[block]{\large\scshape\centering}{\thesection.}{1em}{} % Change the look of the section titles
\titleformat{\subsection}[block]{\large}{\thesubsection.}{1em}{} % Change the look of the section titles
\newcommand{\horrule}[1]{\rule{\linewidth}{#1}} % Create horizontal rule command with 1 argument of height
\usepackage{fancyhdr} % Headers and footers
\pagestyle{fancy} % All pages have headers and footers
\fancyhead{} % Blank out the default header
\fancyfoot{} % Blank out the default footer


%----------------------------------------------------------------------------------------
%       TITLE SECTION
%----------------------------------------------------------------------------------------
\title{\vspace{-15mm}\fontsize{24pt}{10pt}\selectfont\textbf{Data Transformation}} % Article title
\author{
\large
{\textsc{Gahan Saraiya (18MCEC10), Priyanka Bhati (18MCEC02) }}\\[2mm]
%\thanks{A thank you or further information}\\ % Your name
\normalsize \href{mailto:18mcec10@nirmauni.ac.in}{18mcec10@nirmauni.ac.in},\href{mailto:18mcec02@nirmauni.ac.in}{18mcec02@nirmauni.ac.in}\\[2mm] % Your email address
}
\date{}
\hypersetup{
	colorlinks=true,
	linkcolor=blue,
	filecolor=magenta,      
	urlcolor=cyan,
	pdfauthor={Gahan Saraiya},
	pdfcreator={Gahan Saraiya},
	pdfproducer={Gahan Saraiya},
}
%----------------------------------------------------------------------------------------

%----------------------------------------------------------------------------------------
%       SET HEADER AND FOOTER
%----------------------------------------------------------------------------------------
\newcommand\theauthor{Gahan Saraiya}
\newcommand\thesubject{Data Warehousing and Mining}
\renewcommand{\footrulewidth}{0.4pt}% default is 0pt
\fancyhead[C]{Institute of Technology, Nirma University $\bullet$ \monthyeardate\today} % Custom header text
\fancyfoot[LE,LO]{\thesubject}
\fancyfoot[RO,LE]{Page \thepage} % Custom footer text
%----------------------------------------------------------------------------------------

\usepackage[utf8]{inputenc}
\usepackage[english]{babel}
\usepackage[utf8]{inputenc}
\usepackage{fourier} 
\usepackage{array}
\usepackage{makecell}

\renewcommand\theadalign{bc}
\renewcommand\theadfont{\bfseries}
\renewcommand\theadgape{\Gape[4pt]}
\renewcommand\cellgape{\Gape[4pt]}
\newcommand*\tick{\item[\Checkmark]}
\newcommand*\arrow{\item[$\Rightarrow$]}
\newcommand*\fail{\item[\XSolidBrush]}
\usepackage{minted} % for highlighting code sytax
\definecolor{LightGray}{gray}{0.9}

\setminted[text]{
	frame=lines, 
	breaklines,
	baselinestretch=1.2,
	bgcolor=LightGray,
%	fontsize=\small
}

\setminted[python]{
	frame=lines, 
	breaklines, 
	linenos,
	baselinestretch=1.2,
%	bgcolor=LightGray,
%	fontsize=\small
}

\begin{document}
\maketitle % Insert title
\thispagestyle{fancy} % All pages have headers and footers

% REFER :-- https://www.guru99.com/data-mining-tutorial.html#2
\section{Introduction}
Data transformation is the process of converting/transforming data from one form\footnote{i.e. format for different data mining tools} or structure into another form or structure. Data transformation is critical to activities such as data integration and data management. Data transformation can include a range of activities such as: converting data types, data cleaning by removing null values or duplicate data, enrich the data, or perform aggregations, depending on the requirements of the project.

A typical data transformation can be classified in two stages:
\begin{itemize}
	\item Stage 1
	\begin{itemize}
		\item Discovering the data and identify various sources and data types.
		\item Find out the structure of data and data transformations that need to be performed.
		\item Map data to define how individual fields are mapped, modified, joined, filtered, and aggregated.
	\end{itemize}
	\item Stage 2
	\begin{itemize}
		\item Extract data from the original source. 
		\\ The range of sources can vary from structured sources like databases to non-structured sources such as streaming sources.
		\item Do the transformation
		\\ i.e. aggregating values, changing date formats, editing text or joining attributes.
		\item Migrate this transformed data to data warehouse.			
	\end{itemize}
\end{itemize}

\section{Transformation of Data}
\subsection{Data Discovery}
We have taken a dataset as described in Section \ref{section:experiments}.\ref{sample-data} for which we need to perform knowledge gain of user mobility with correlation between locations.

\subsection{Data Structure}
Data is a raw data of multiple text files (one file per taxi id) containing records of booking of taxi with timestamp and latitude and longitude of pickup location.

\subsection{Extracting Data}
To perform operation we will first combine these multiple data files into one single file containing all the record of \verb|taxi-id, latitude, longitude and timestamp|

\subsection{Performing Transformation}
The two attributes in dataset \ref{section:experiments}.\ref{sample-data}, latitude and longitude are concatenate together to form unique area with method known as geohash described in Section \ref{section:experiments}.\ref{geohash}.

This will generate a file (as in \ref{section:experiments}.\ref{geo-hashed}) containing \verb|taxi-id, geohash and timestamp|

\section{Data Visualization}
Now we need to compute correlation between locations from our transformed data \ref{section:experiments}.\ref{geo-hashed}.

To compute correlation we simply create dissimilarity between two records (which is considered as the timestamp).

Visualization result is obtained as in Section \ref{section:experiments}.\ref{result} which contains \verb|geohash, geohash and dissimilarity|.



\section{Experiment}\label{section:experiments}
\subsection{Dataset}\label{sample-data}
For the experiments we have taken a dataset of user mobility which contains data attributes as described below:
\begin{table}[H]
	\centering
	\renewcommand{\arraystretch}{1.5}
	\begin{tabular}{l | l | l}
		Attribute & Data-type & Description
		\\ \hline \hline
		Taxi-id & Numerical (Integer) & unique id of each taxi
		\\ \hline
		Latitude & Discrete & latitude of pickup location
		\\ \hline
		Longitude & Discrete & longitude of pickup location
		\\ \hline
		Timestamp & date-time & time-stamp when taxi booked
	\end{tabular}
	\caption{Description of Dataset}
\end{table}

\subsubsection{Chunk 1}\label{chunk-1}
\inputminted[firstline=0,lastline=10]{text}{../file1.txt}
\subsubsection{Chunk 2}\label{chunk-2}
\inputminted[firstline=2050,lastline=2060]{text}{../file1.txt}

\subsection{Geohash}\label{geohash}
The Geohash is the method for encoding regions with specific precision of latitude and longitude. 

Geohashes offer properties like arbitrary precision and the possibility of gradually removing characters from the end of the code to reduce its size (and gradually lose precision).

For example, the coordinate pair \verb|57.64911,10.40744| (near the tip of the peninsula of Jutland, Denmark) produces a slightly shorter hash of \verb|u4pruydqqvj|.

Implementation of Geohash is described below:
\subsubsection{Implementation}
\inputminted{python}{../geohash.py}

\subsubsection{Data File after applying geohash}\label{geo-hashed}
\subsubsection{Chunk 1}\label{hash-chunk-1}
\inputminted[firstline=0,lastline=10]{text}{../file2.txt}
\subsubsection{Chunk 2}\label{hash-chunk-2}
\inputminted[firstline=2050,lastline=2060]{text}{../file2.txt}

\subsection{Implementation of Utilities for parsing data}\label{data-read}
\inputminted{python}{../simulator.py}

\subsection{Implementation of storing and transforming data}\label{data-transformation}
\inputminted{python}{../main.py}

\subsubsection{Result}\label{result}
\begin{minted}{text}
2-2,  uzurg,  uzuru, -1411.0 
2-2,  uzurg,  uzuru, -1508.0 
2-2,  uzurg,  uzuru, -1713.0 
2-2,  uzurg,  uzuru, -1810.0 
2-2,  uzurg,  uzuru, -1810.0 
2-2,  uzurg,  uzuru, -2111.0 
2-2,  uzurg,  uzurf, -10466.0 
2-2,  uzurg,  uzurf, -10723.0 
2-2,  uzurg,  uzurf, -10768.0 
2-2,  uzurg,  uzurf, -11025.0 
\end{minted}
$ \vdots $
\begin{minted}{text}
2-4,  uzurg,  uzuru, -397746.0 
2-4,  uzurg,  uzuru, -398347.0 
2-4,  uzurg,  uzuru, -398947.0 
2-4,  uzurg,  uzuru, -399546.0 
2-4,  uzurg,  uzuru, -400146.0 
2-4,  uzurg,  uzuru, -400746.0 
2-4,  uzurg,  uzuru, -401946.0 
2-4,  uzurg,  uzuru, -402546.0 
2-4,  uzurg,  uzuru, -403146.0 
2-4,  uzurg,  uzuru, -403745.0 
2-4,  uzurg,  uzuru, -404345.0 
2-4,  uzurg,  uzuru, -404945.0 
2-4,  uzurg,  uzuru, -405546.0 
2-4,  uzurg,  uzuru, -406146.0 
2-4,  uzurg,  uzuru, -406745.0 
\end{minted}
$ \vdots $
\begin{minted}{text}
5-5,  uzuru,  uzurf, -229832.0 
5-5,  uzuru,  uzurf, -230432.0 
5-5,  uzuru,  uzurf, -231032.0 
5-5,  uzuru,  uzurf, -231632.0 
5-5,  uzuru,  uzurf, -232232.0 
5-5,  uzuru,  uzurf, -232832.0 
5-5,  uzuru,  uzurf, -233431.0 
5-5,  uzuru,  uzurf, -234031.0 
5-5,  uzuru,  uzurf, -234631.0 
5-5,  uzuru,  uzurf, -235231.0 
\end{minted}

\begin{table}[H]
	\centering
	\renewcommand{\arraystretch}{1.5}
	\begin{tabular}{l | l | l}
		Attribute & Data-type & Description
		\\ \hline \hline
		taxi-id-taxi-id & string & ids of two taxis separated by hyphen
		\\ \hline
		Area & geohash & hashed location value
		\\ \hline
		Area & geohash & hashed location value
		\\ \hline
		Dissimilarity & Float value & Float value containing difference of epoch timestamp
	\end{tabular}
	\caption{Description of final result: dissimilarity between location}
\end{table}

\section{Conclusion}
The technique geohash which we have used for data transformation gives the dissimilarity with
greater precision. But this precision leads to higher computation cost which is the major drawback.
So to reduce the computation cost the value of precision needs to be decreased to reduce
computational cost.

\end{document}

%%%%%%%%%%%%%%%%%%%%%%%%%%%%%%%%%%%%%%%%%
% Journal Article
% Data Warehouse and Mining
% Practical 1: Data Domain selection and Identification of Characteristics of selected Dataset of different formats
%
% web-guide : https://sites.google.com/a/nirmauni.ac.in/3cs1e23---data-warehousing-and-mining/home/academic-docs/list-of-practicals/practical---1?pli=1
% Gahan M. Saraiya
% 18MCEC10
%
%%%%%%%%%%%%%%%%%%%%%%%%%%%%%%%%%%%%%%%%%
%----------------------------------------------------------------------------------------
%       PACKAGES AND OTHER DOCUMENT CONFIGURATIONS
%----------------------------------------------------------------------------------------
\documentclass[paper=letter, fontsize=12pt]{article}
\usepackage[english]{babel} % English language/hyphenation
\usepackage{amsmath,amsfonts,amsthm} % Math packages
\usepackage[utf8]{inputenc}
\usepackage{xcolor}
\usepackage{float}
\usepackage{lipsum} % Package to generate dummy text throughout this template
\usepackage{blindtext}
\usepackage{graphicx} 
\usepackage{caption}
\usepackage{subcaption}
\usepackage[sc]{mathpazo} % Use the Palatino font
\usepackage[T1]{fontenc} % Use 8-bit encoding that has 256 glyphs
\usepackage{bbding}  % to use custom itemize font
\linespread{1.05} % Line spacing - Palatino needs more space between lines
\usepackage{microtype} % Slightly tweak font spacing for aesthetics
\usepackage[hmarginratio=1:1,top=32mm,columnsep=20pt]{geometry} % Document margins
\usepackage{multicol} % Used for the two-column layout of the document
%\usepackage[hang, small,labelfont=bf,up,textfont=it,up]{caption} % Custom captions under/above floats in tables or figures
\usepackage{booktabs} % Horizontal rules in tables
\usepackage{float} % Required for tables and figures in the multi-column environment - they need to be placed in specific locations with the [H] (e.g. \begin{table}[H])
\usepackage{hyperref} % For hyperlinks in the PDF
\usepackage{lettrine} % The lettrine is the first enlarged letter at the beginning of the text
\usepackage{paralist} % Used for the compactitem environment which makes bullet points with less space between them
\usepackage{abstract} % Allows abstract customization
\renewcommand{\abstractnamefont}{\normalfont\bfseries} % Set the "Abstract" text to bold
\renewcommand{\abstracttextfont}{\normalfont\small\itshape} % Set the abstract itself to small italic text
\usepackage{titlesec} % Allows customization of titles

\usepackage{makecell}
\usepackage{longtable}
\renewcommand\thesection{\Roman{section}} % Roman numerals for the sections
\renewcommand\thesubsection{\Roman{subsection}} % Roman numerals for subsections
%----------------------------------------------------------------------------------------
%       DATE FORMAT
%----------------------------------------------------------------------------------------
\usepackage{datetime}
\newdateformat{monthyeardate}{\monthname[\THEMONTH], \THEYEAR}
%----------------------------------------------------------------------------------------

\titleformat{\section}[block]{\large\scshape\centering}{\thesection.}{1em}{} % Change the look of the section titles
\titleformat{\subsection}[block]{\large}{\thesubsection.}{1em}{} % Change the look of the section titles
\newcommand{\horrule}[1]{\rule{\linewidth}{#1}} % Create horizontal rule command with 1 argument of height
\usepackage{fancyhdr} % Headers and footers
\pagestyle{fancy} % All pages have headers and footers
\fancyhead{} % Blank out the default header
\fancyfoot{} % Blank out the default footer


%----------------------------------------------------------------------------------------
%       TITLE SECTION
%----------------------------------------------------------------------------------------
\title{\vspace{-15mm}\fontsize{24pt}{10pt}\selectfont\textbf{Practical 1: Understanding Characteristics of Data Sources}} % Article title
\author{
\large
{\textsc{Gahan Saraiya (18MCEC10), Priyanka Bhati (18MCEC02) }}\\[2mm]
%\thanks{A thank you or further information}\\ % Your name
\normalsize \href{mailto:18mcec10@nirmauni.ac.in}{18mcec10@nirmauni.ac.in},\href{mailto:18mcec02@nirmauni.ac.in}{18mcec02@nirmauni.ac.in}\\[2mm] % Your email address
}
\date{}
\hypersetup{
	colorlinks=true,
	linkcolor=blue,
	filecolor=magenta,      
	urlcolor=cyan,
	pdfauthor={Gahan Saraiya},
	pdfcreator={Gahan Saraiya},
	pdfproducer={Gahan Saraiya},
}
%----------------------------------------------------------------------------------------

%----------------------------------------------------------------------------------------
%       SET HEADER AND FOOTER
%----------------------------------------------------------------------------------------
\newcommand\theauthor{Gahan Saraiya}
\newcommand\thesubject{Data Warehousing and Mining}
\renewcommand{\footrulewidth}{0.4pt}% default is 0pt
\fancyhead[C]{Institute of Technology, Nirma University $\bullet$ \monthyeardate\today} % Custom header text
\fancyfoot[LE,LO]{\thesubject}
\fancyfoot[RO,LE]{Page \thepage} % Custom footer text
%----------------------------------------------------------------------------------------

\usepackage[utf8]{inputenc}
\usepackage[english]{babel}
\usepackage[utf8]{inputenc}
\usepackage{fourier} 
\usepackage{array}
\usepackage{makecell}

\renewcommand\theadalign{bc}
\renewcommand\theadfont{\bfseries}
\renewcommand\theadgape{\Gape[4pt]}
\renewcommand\cellgape{\Gape[4pt]}
\newcommand*\tick{\item[\Checkmark]}
\newcommand*\arrow{\item[$\Rightarrow$]}
\newcommand*\fail{\item[\XSolidBrush]}
\usepackage{minted} % for highlighting code sytax
\definecolor{LightGray}{gray}{0.9}

\setminted[text]{
	frame=lines, 
	breaklines,
	baselinestretch=1.2,
	bgcolor=LightGray,
%	fontsize=\small
}

\setminted[python]{
	frame=lines, 
	breaklines, 
	linenos,
	baselinestretch=1.2,
%	bgcolor=LightGray,
%	fontsize=\small
}

\begin{document}
\maketitle % Insert title
\thispagestyle{fancy} % All pages have headers and footers

% REFER :-- https://www.guru99.com/data-mining-tutorial.html#2
\section{AIM}
Data Domain selection and Identification of Characteristics of selected Dataset of different formats.

\section{Introduction}
Data mining can be performed on following types of data 
\begin{itemize}
	\item Text databases
	\item Object-oriented and object-relational databases
	\item Heterogeneous and legacy databases
	\item Relational databases
	\item Transnational and Spatial databases
	\item Multimedia database
	\item Text mining and Web mining
	\item Data warehouses
	\item Advanced DB and information repositories
	
\end{itemize}
This practical aims towards data understanding to check that business and data-mining goals are established as well as to make it appropriate for data-mining goals if applicable.

\subsection{Domain Identification}
Data Mining can be used in diverse industries such as Retail market, Communications, Insurance, Education, Manufacturing, E-commerce, Banking, Bio-informatics, Crime Investigation etc.
To achieve this goal this document  targets on the mining process on Retail market .
\paragraph{} Data Mining techniques help \textbf{retail} malls and grocery stores to identify and arrange most sellable items in the most attentive positions. It helps store owners to comes up with various offers which encourages customers to spend more. 
Also Note that the same scenario is also applicable to local retail stores as well as e-commerce stores with slightly differ in presenting the recommendation to customers.

\section{Extracting Data}
Techniques for Extracting Data:
\begin{itemize}
    \item Extracting from Database
    \item Extracting through APIs
    \item Extracting via Web Scraping
    \item Public Dataset
\end{itemize}
In this experiment the data extraction is performed from sale details of various retail market store where the data warehouse is of dimension $ 537577 \times 12 $ i.e. contains twelve $ 12 $ different attributes and $ 5,37,577 $ data records.

For the simplicity cities are categorized in to three different category $ A, B, C $ and the products of sale is classified in to three different category.

\section{Characterization of dataset}
From the various Retail market events this experiment aims to a single event of data mining for \textbf{Black Friday Sale} records from various retail markets over cities categorized.

\begin{itemize}
	\item Dataset contain 5,37,577 entries of Black Friday Sale in a retail store.
\end{itemize}
\subsection{Sample Dataset}
%\inputminted[firstline=15]{text}{../src/BlackFriday.csv}
\begin{minted}[fontsize=\small]{text}
User_ID, Product_ID, Gender, Age, Occupation, City_Category, Stay_In_Current_City_Years, Marital_Status, Product_Category_1, Product_Category_2, Product_Category_3, Purchase
1000001,P00069042,F,0-17,10,A,2,0,3,,,8370
1000001,P00248942,F,0-17,10,A,2,0,1,6,14,15200
1000001,P00087842,F,0-17,10,A,2,0,12,,,1422
1000001,P00085442,F,0-17,10,A,2,0,12,14,,1057
1000002,P00285442,M,55+,16,C,4+,0,8,,,7969
1000003,P00193542,M,26-35,15,A,3,0,1,2,,15227
1000004,P00184942,M,46-50,7,B,2,1,1,8,17,19215
1000004,P00346142,M,46-50,7,B,2,1,1,15,,15854
1000004,P0097242,M,46-50,7,B,2,1,1,16,,15686
1000005,P00274942,M,26-35,20,A,1,1,8,,,7871
1000005,P00251242,M,26-35,20,A,1,1,5,11,,5254
1000005,P00014542,M,26-35,20,A,1,1,8,,,3957
1000005,P00031342,M,26-35,20,A,1,1,8,,,6073
1000005,P00145042,M,26-35,20,A,1,1,1,2,5,15665
1000006,P00231342,F,51-55,9,A,1,0,5,8,14,5378
1000006,P00190242,F,51-55,9,A,1,0,4,5,,2079
1000006,P0096642,F,51-55,9,A,1,0,2,3,4,13055
1000006,P00058442,F,51-55,9,A,1,0,5,14,,8851
1000007,P00036842,M,36-45,1,B,1,1,1,14,16,11788
1000008,P00249542,M,26-35,12,C,4+,1,1,5,15,19614
1000008,P00220442,M,26-35,12,C,4+,1,5,14,,8584
1000008,P00156442,M,26-35,12,C,4+,1,8,,,9872
1000008,P00213742,M,26-35,12,C,4+,1,8,,,9743
1000008,P00214442,M,26-35,12,C,4+,1,8,,,5982
1000008,P00303442,M,26-35,12,C,4+,1,1,8,14,11927
\end{minted}

\subsection{Property of Dataset}
\begin{table}[H]
	\centering
	\begin{tabular}{l  l  l  l}
		Field & Type & Remarks & Non-null Entries
		\\ \hline \hline
		User\_ID & ID & Identify User by this ID & 537577
		\\
		Product\_ID & String & Identify Product & 537577
		\\
		Gender & String &  & 537577
		\\
		Age & String & Range of Age & 537577
		\\
		Occupation & Integer & Generalized occupation by integer & 537577
		\\
		City\_Category & String & Classifies City & 537577
		\\
		Stay\_In\_Current\_City\_Years & String & Classify duration of stay in city & 537577
		\\
		Marital\_Status & Integer &  & 537577
		\\
		Product\_Category\_1 & Integer &  & 537577
		\\
		Product\_Category\_2 & Integer &  & 370591
		\\
		Product\_Category\_3 & Integer &  & 164278
		\\
		Purchase & Integer &  & 537577
	\end{tabular}
    \caption{Dataset Property}
    \label{table:data-observation}
\end{table}

After merging data it is observed that for Product Category 2 and 3 there are missing values as described in Table \ref{table:data-observation}.

\section{Querying dataset}
\include{notebook}


\section{Intuition from dataset : Possible problems and their solution}
\subsection{Where to open new store?}
\begin{enumerate}
	\item Visualize data of trends for relative purchase and city category
	\item Gather relative association between purchase and city
	\item deduce location of opening store based on purchase amount and number of product sale
\end{enumerate}

\subsection{Marketing of new store $ \dots $}
\begin{enumerate}
	\item Gather trends of product category sales in city
	\item Gather trends of product category sales among age groups
	\item Gather trends of product category sales among gender
	\item Gather trends of product category sales for different marital status
	\item Plot relation between city, age groups, gender and marital status and market the sale of trending products
\end{enumerate}

\subsection{Predicting trends of product among different age group}
One can able to determine which product categories are popular among which area of city category so that that the retail store can focus on purchase/resell of product with more relevant category based on different gender, marital status and age group.



\end{document}
